\documentclass[aps,preprint,12pt]{revtex4-1}
%\documentclass[aip,preprint,12pt]{revtex4-1}
\usepackage{multirow}
\usepackage{amssymb}
\usepackage{amsmath}
\usepackage{graphicx}
\usepackage{algorithmic}
\usepackage{algorithm}
\usepackage{color}


\begin{document}

\title{Total energy and SCF in PWSCF}
\author{ Honghui Shang}
\email[Electronic address:\;]{shang@fhi-berlin.mpg.de}

\affiliation{Fritz-Haber-Institut der Max-Planck-Gesellschaft,
  Faradayweg 4-6, D-14195 Berlin-Dahlem, Germany}

\date{\today}

\begin{abstract}
The total energy and SCF equation using plane wave basis sett is written here. I have also compared PW basis set with AO basis set. So we can see the difference between 'near-free' electron approximation and tight-binding.   We follow the Kohn-Sham approach \cite{DFT_KS} to
derive our equations.  The Kohn-Sham ansatz( A mathematical assumption, especially about the form of an 
unknown function, which is made in order to facilitate solution of an equation or other problem) is to replace the difficult interaction many-body system with this Hamiltonian 
\begin{equation}
\hat{H}=-\dfrac{1}{2}\sum_{i}\bigtriangledown^2_{i}-\sum_{i}\sum_{I}\dfrac{Z_I}{|\mathbf{r_i}-\mathbf{R_I}|}
+\dfrac{1}{2}\sum_{i}\sum_{j\neq i}\dfrac{1}{|\mathbf{r_i}-\mathbf{r_j}|}
+\dfrac{1}{2}\sum_{I}\sum_{J\neq I}\dfrac{Z_I Z_J}{|\mathbf{R_I}-\mathbf{R_J}|}
\end{equation} into a independent-particle problem. What they have been done is to define a ground state
energy $E_{KS}$, and then derive a Kohn-Sham Sch\"{o}dinger-like equation using Lagrange multipliers or
Rayleigh-Ritz principle.  In this paper we will show the detailed formula derivation in PW basis set.
\end{abstract}
\maketitle


\section{METHODS}


\subsection{Total energy expressions are the same in both basis set}
Kohn-Sham wrote the total energy of a many-body system as:
\[
E_{KS}=-\dfrac{1}{2}\sum_{i}<\phi_i|\nabla^2|\phi_i>-\int {\rho(\mathbf{r})  \sum_{I}\dfrac{Z_{I}}{|\mathbf{r}-\mathbf{R}_{I}|}  d\mathbf{r}}+
\]
\begin{equation}
 \dfrac{1}{2}\int \int {\dfrac{\rho(\mathbf{r}) \rho(\mathbf{r'}) }{|\mathbf{r}-\mathbf{r'} |}  d\mathbf{r}  d\mathbf{r'}} +\dfrac{1}{2}\sum_{I}\sum_{J}{\dfrac{Z_{I} Z_{J}}{|\mathbf{R}_{I}-\mathbf{R}_{J} |} }+ E_{xc}(\rho)
\label{eq:E_KS}
\end{equation}
\begin{equation}
= T+E_{ext}+E_{hartree}+E_{IJ}+E_{xc}
\end{equation}

\subsection{crystal-orbital-SCF equations are the same in both basis set}
So all the many-body effect go to $E_{xc}(\rho)$ term. Using Lagrange multipliers, we have 
\begin{equation}
\dfrac{\delta [E_{KS}-\sum_{i}\epsilon_{i}(<\phi_i|\phi_i> -1)] }
{\delta \phi_i} =  0 =
\end{equation}
\begin{equation}
\dfrac{\delta T}{\delta \phi_i} + 
[\dfrac{\delta E_{ext}}{\delta \rho(r)} + \dfrac{\delta E_{hartree}}{\delta \rho(r)} +  \dfrac{\delta E_{xc}}{\delta \rho(r)} ] 
\dfrac{\delta \rho(r)}{\delta \phi_i}-\epsilon_{i} \phi_i = 
\end{equation}
\begin{equation}
 -\dfrac{1}{2}\nabla^2 \phi_i + 
[V_{ext}(r)+V_{hartree}+V_{xc}] \phi_i -\epsilon_{i} \phi_i = 0
\label{eq:ks-equation}
\end{equation}

So we get Kohn-Sham Sch\"{o}dinger-like equation in Eq. \ref{eq:ks-equation}.


\subsection{SCF equations in PW basis set}
In solid state physics, Bloch theorem stays that all
the wave function should be Bloch wave function. We can
express 





\section{CONCLUSIONS}

\begin{acknowledgments}

\end{acknowledgments}


\begin{thebibliography} {99}


\bibitem{DFT_KS} W. Kohn and L. Sham, Phys. Rev. {\bf 140}, 1133 (1965).
\bibitem{DFT_HK} P. Hohenberg and W. Kohn, Phys. Rev. B {\bf 136}, 864 (1964).

\bibitem{Parr} R. G. Parr and W. Yang, \emph{Density Functional Theory of Atoms and Molecules }(Oxford University Press, New York,1989)

\end{thebibliography}



\end{document}
