\documentclass[aps,preprint,12pt]{revtex4-1}
%\documentclass[aip,preprint,12pt]{revtex4-1}
\usepackage{multirow}
\usepackage{amssymb}
\usepackage{amsmath}
\usepackage{graphicx}
\usepackage{algorithmic}
\usepackage{algorithm}
\usepackage{color}


\begin{document}

\title{HSE06 functional in siesta}
\author{ Honghui Shang}
\email[Electronic address:\;]{shang@fhi-berlin.mpg.de}

\affiliation{Fritz-Haber-Institut der Max-Planck-Gesellschaft,
  Faradayweg 4-6, D-14195 Berlin-Dahlem, Germany}

\date{\today}

\begin{abstract}
The total energy and force equations for HSE06 functional in siesta is written here. By this way, further
 extension (mp2, hessian) can be done in my coming work. Here we follow the Kohn-Sham approach \cite{DFT_KS} to
derive our equations.  The Kohn-Sham ansatz( A mathematical assumption, especially about the form of an 
unknown function, which is made in order to facilitate solution of an equation or other problem) is to replace the difficult interaction many-body system with this Hamiltonian 
\begin{equation}
\hat{H}=-\dfrac{1}{2}\sum_{i}\bigtriangledown^2_{i}-\sum_{i}\sum_{I}\dfrac{Z_I}{|\mathbf{r_i}-\mathbf{R_I}|}
+\dfrac{1}{2}\sum_{i}\sum_{j\neq i}\dfrac{1}{|\mathbf{r_i}-\mathbf{r_j}|}
+\dfrac{1}{2}\sum_{I}\sum_{J\neq I}\dfrac{Z_I Z_J}{|\mathbf{R_I}-\mathbf{R_J}|}
\end{equation} into a independent-particle problem. What they have been done is to define a ground state
energy $E_{KS}$, and then derive a Kohn-Sham Sch\"{o}dinger-like equation using Lagrange multipliers or
Rayleigh-Ritz principle.  In this paper we will show the detailed formula derivation.

\end{abstract}
\maketitle


\section{METHODS}
\subsection{HSE total energy}
Kohn-Sham wrote the total energy of a many-body system as:
\[
E_{KS}=-\dfrac{1}{2}\sum_{i}<\phi_i|\nabla^2|\phi_i>-\int {\rho(\mathbf{r})  \sum_{I}\dfrac{Z_{I}}{|\mathbf{r}-\mathbf{R}_{I}|}  d\mathbf{r}}+
\]
\begin{equation}
 \dfrac{1}{2}\int \int {\dfrac{\rho(\mathbf{r}) \rho(\mathbf{r'}) }{|\mathbf{r}-\mathbf{r'} |}  d\mathbf{r}  d\mathbf{r'}} +\dfrac{1}{2}\sum_{I}\sum_{J}{\dfrac{Z_{I} Z_{J}}{|\mathbf{R}_{I}-\mathbf{R}_{J} |} }+ E_{xc}(\rho)
\label{eq:E_KS}
\end{equation}
\begin{equation}
= T+E_{ext}+E_{hartree}+E_{IJ}+E_{xc}
\end{equation}
So all the many-body effect go to $E_{xc}(\rho)$ term. Using Lagrange multipliers, we have 
\begin{equation}
\dfrac{\delta [E_{KS}-\sum_{i}\epsilon_{i}(<\phi_i|\phi_i> -1)] }
{\delta \phi_i} =  0 =
\end{equation}
\begin{equation}
\dfrac{\delta T}{\delta \phi_i} + 
[\dfrac{\delta E_{ext}}{\delta \rho(r)} + \dfrac{\delta E_{hartree}}{\delta \rho(r)} +  \dfrac{\delta E_{xc}}{\delta \rho(r)} ] 
\dfrac{\delta \rho(r)}{\delta \phi_i}-\epsilon_{i} \phi_i = 
\end{equation}
\begin{equation}
 -\dfrac{1}{2}\nabla^2 \phi_i + 
[V_{ext}(r)+V_{hartree}+V_{xc}] \phi_i -\epsilon_{i} \phi_i = 0
\label{eq:ks-equation}
\end{equation}

So we get Kohn-Sham Sch\"{o}dinger-like equation in Eq. \ref{eq:ks-equation}.

In siesta, using psedopotential, so the $E_{KS}$ in siesta is written as Eq. 53 of siesta-2002 paper, again, it equals $E_{KS}$ we get before:
\begin{equation}
E_{KS}= T + E_{ext}+ E_{hartree} + E_{IJ}+E_{xc} 
\end{equation} 

\textcolor{blue}{Here I gave a derivation of siesta's Eq(53):} 
\[
E_{KS-siesta}=-\dfrac{1}{2}\sum_{i}<\phi_i|\nabla^2|\phi_i> +\sum_{i}<\phi_i|V^{KB}|\phi_i>+\int {\rho^{pseudo}(\mathbf{r})  \sum_{I}V_I^{local}(r)  d\mathbf{r}}+
\]
\begin{equation}
\dfrac{1}{2}\int \int {\dfrac{\rho^{pseudo}(\mathbf{r}) \rho^{pseudo}(\mathbf{r'}) }{|\mathbf{r}-\mathbf{r'} |}  d\mathbf{r}  d\mathbf{r'}} +\dfrac{1}{2}\sum_{I}\sum_{J}{\dfrac{Z^{pseudo}_{I} Z^{pseudo}_{J}}{|\mathbf{R}_{I}-\mathbf{R}_{J} |} }+ E_{xc}(\rho)
\label{eq:E_KS_siesta}
\end{equation}
here all the $\rho^{pseudo}(r)$ and $Z^{pseudo}_I$ and valence contribution, because we have already used pseudo-potential, in which way we only see valence electrons in system. 
$V_I^{local}(r)$ looks like $-Z^{pseudo}_I/|r-R_I|$ in far distance, but do not diverge near nuclear position. 
\begin{equation}
V_I^{local}(r) \approx -Z^{pseudo}_I/|r-R_I|
\end{equation}


\begin{equation}
U_{IJ}^{local}(R)=\int{ V_I^{local}(r) {\rho}_J^{local}(r-R)} dr 
\end{equation}

\begin{equation}
\dfrac{1}{2}\sum_{I}\sum_{J\neq I}{\dfrac{Z^{pseudo}_{I} Z^{pseudo}_{J}}{|\mathbf{R}_{I}-\mathbf{R}_{J} |} }=\dfrac{1}{2}\sum_{I}\sum_{J}U_{IJ}^{local}(R_{IJ})-\dfrac{1}{2}\sum_{I}U_{II}^{local}(0) + \dfrac{1}{2}\sum_{I}\sum_{J\neq I} \delta U_{IJ}^{local}(R_{IJ})
\end{equation}


\begin{equation}
\dfrac{1}{2}\int \int {\dfrac{\rho^{pseudo}(\mathbf{r}) \rho^{pseudo}(\mathbf{r'}) }{|\mathbf{r}-\mathbf{r'} |}  
d\mathbf{r}  d\mathbf{r'} } =  \dfrac{1}{2}\int \int {\dfrac{ [\rho^{atom}(r)+\delta\rho(r) ] [\rho^{atom}(r)+\delta\rho(r) ] }{|\mathbf{r}-\mathbf{r'} |}  
d\mathbf{r}  d\mathbf{r'} }  
\end{equation}

\begin{equation}
V^{NA}(r)=V^{local}(r)+V^{atom}(r)
\end{equation}

So Eq.(\ref{eq:E_KS_siesta}) can be written as:
\[
E_{KS-siesta}=
\]
\[
-\dfrac{1}{2}\sum_{i}<\phi_i|\nabla^2|\phi_i> +\sum_{i}<\phi_i|V^{KB}|\phi_i>+
\int{[\rho^{atom}(r)+\delta\rho(r) ] V^{local}(r)  d\mathbf{r}}+E_{xc}(\rho)
\]
\[
+\dfrac{1}{2}\int \int {\dfrac{ [\rho^{atom}(r)+\delta\rho(r) ] [\rho^{atom}(r)+\delta\rho(r) ] }{|\mathbf{r}-\mathbf{r'} |}  
d\mathbf{r}  d\mathbf{r'} }  
\]
\[
+\dfrac{1}{2}\sum_{I}\sum_{J}U_{IJ}^{local}(R_{IJ})-\dfrac{1}{2}\sum_{I}U_{II}^{local}(0) + \dfrac{1}{2}\sum_{I}\sum_{J\neq I} \delta U_{IJ}^{local}(R_{IJ})
\]

\[
= -\dfrac{1}{2}\sum_{i}<\phi_i|\nabla^2|\phi_i> +\sum_{i}<\phi_i|V^{KB}|\phi_i>
+E_{xc}(\rho)+\dfrac{1}{2}\int\int {\dfrac{ \delta \rho(r') \delta \rho(r)}{|r-r'|} dr' dr} 
\]
\[
-\dfrac{1}{2}\sum_{I}U_{II}^{local}(0) + \dfrac{1}{2}\sum_{I}\sum_{J\neq I} \delta U_{IJ}^{local}(R_{IJ})
\] 
\[
+\int{[V^{local}(r)+V^{atom}(r)]\delta \rho(r) dr   } ==> \int{V^{NA}(r)\delta \rho(r) dr   }
\]
\begin{equation}
+\dfrac{1}{2}\sum_{I}\sum_{J}\int{[V^{local}(r)+V^{atom}(r)][\rho^{local}+\rho^{atom} ] dr   }==>  \dfrac{1}{2}\sum_{I}\sum_{J} U_{IJ}(R_{IJ})
\label{eq:E_KS_siesta_my}
\end{equation}
So it is the same as Eq(53) in siesta's paper. 
\textcolor{red}{Here the idea is the same as Ewald method}, because here $\rho^{local}$ is just like Gaussian charge in Ewald, and
then the $V^{NA}$ is short ranged and calculated in real space and  $\delta V(r)$  is calculated in moment space.   






So for HSE total energy, we only add$ E_{HSE}$ to $E_{xc}$ part. 
The expression for HSE06 is given by: 
\begin{equation}
 E_{xc}^{HSE}=\dfrac{1}{4}E_{x}^{SR-HF}(\omega)+\dfrac{3}{4}E_{x}^{SR-PBE}(\omega)
+E_{x}^{LR-PBE}(\omega)+E_{c}^{PBE}
\end{equation} 

where  $\omega$ =$0.11 Bohr^{-1}$.



For un-spin-polarized systems (nspin=1), the Hartree-Fock exchange matrix element is defined as (here the 1/2 is coming from exchange interaction, when we use slater-determinant to get HF total energy): 
\begin{equation}
  [V^{X}]_{\mu\lambda}^{\mathbf{G}}=-\frac{1}{2}\sum_{\nu\sigma}\sum_{\mathbf{N,H}}P_{\nu\sigma}^\mathbf{H-N}\mathbf{[(\chi_{\mu}^{0}\chi_{\nu}^{N}|\chi_{\lambda}^{G}\chi_{\sigma}^{H})]}
%  \label{eq:vx}
\end{equation}
where $\mathbf{G}$, $\mathbf{N}$, and $\mathbf{H}$ represent different unit cells.


So we can get  Hartree-Fock exchange energy , here 1/4 is because exchange energy is always 1/2 of hartree energy (3rd term in Eq. \ref{eq:E_KS}), so $\dfrac{1}{2}*\dfrac{1}{2}=\dfrac{1}{4} $
 \begin{equation}
  E^{HFX}=-\frac{1}{4}\sum_{\mu \lambda}\sum_{\mathbf{G}}{ P_{\mu\lambda}^{\mathbf{G}} }  \sum_{\nu\sigma}\sum_{\mathbf{N,H}}
 P_{\nu\sigma}^\mathbf{H-N}\mathbf{[(\chi_{\mu}^{0}\chi_{\nu}^{N}
  |\chi_{\lambda}^{G}\chi_{\sigma}^{H})]}
\end{equation}



\subsection{HSE total force}
The PBE part force is calculate directly with siesta, we only write Hartree-Fock Exchange part for HSE force here.


For un-spin-polarized systems (nspin=1), the Gradient is divided into two terms:
\begin{equation}
 \dfrac{\partial{E_{HFX}}}{\partial{R_{I}}}
=-\frac{1}{2}\sum_{\mu\lambda}\sum_{\mathbf{G}}\dfrac{P_{\mu\lambda}^{\mathbf{G}}}{\partial{R_{I}}}\sum_{\nu\sigma}\sum_{\mathbf{N,H}}P_{\nu\sigma}^\mathbf{H-N}\mathbf{[(\chi_{\mu}^{0}\chi_{\nu}^{N}|\chi_{\lambda}^{G}\chi_{\sigma}^{H})]}
\end{equation}
\[
 -\frac{1}{4}\sum_{\mu\lambda}\sum_{\mathbf{G}}P_{\mu\lambda}^{\mathbf{G}}\sum_{\nu\sigma}\sum_{\mathbf{N,H}}P_{\nu\sigma}^\mathbf{H-N}\mathbf{ \dfrac{\partial{(\chi_{\mu}^{0}\chi_{\nu}^{N}|\chi_{\lambda}^{G}\chi_{\sigma}^{H})}}{\partial{R_{I}}} }
\]


The first term can be calculated in the orthogonalization force:
\begin{equation}
 \sum_{\mu\nu}{F_{\mu\nu}\dfrac{\partial{P_{\mu\nu}}}{\partial{R_{I}}}}=
-\sum_{\mu\nu}{E_{\mu\nu}\dfrac{\partial{S_{\mu\nu}}}{\partial{R_{I}}}}
\end{equation}
where
\[
 E_{\mu\nu}=\sum_{i}{c_{\mu i}c_{\nu i}n_i\varepsilon{i} }
\]


The second term need the gradient of ERIs. In the following, we will deal with this term: 
\[
 F_{\mathbf{R_I}}=\frac{1}{4}\sum_{\mu\lambda}\sum_{\mathbf{G}}P_{\mu\lambda}^{\mathbf{G}}\sum_{\nu\sigma}\sum_{\mathbf{N,H}}P_{\nu\sigma}^\mathbf{H-N}\mathbf{ \dfrac{\partial{(\chi_{\mu}^{0}\chi_{\nu}^{N}|\chi_{\lambda}^{G}\chi_{\sigma}^{H})}}{\partial{R_{I}}} }
\]
\[
 =\frac{1}{4}\sum_{\mu\lambda}\sum_{\mathbf{G}}P_{\mu\lambda}^{\mathbf{G}}\sum_{\nu\sigma}\sum_{\mathbf{N,H}}P_{\nu\sigma}^\mathbf{H-N}
\]
\begin{equation}
\times\mathbf{[ (\dfrac{\chi_{\mu}^{0}}{\partial{R_{I}}}\chi_{\nu}^{N}|\chi_{\lambda}^{G}\chi_{\sigma}^{H})
+(\chi_{\mu}^{0}\dfrac{\chi_{\nu}^{N}}{\partial{R_{I}}}|\chi_{\lambda}^{G}\chi_{\sigma}^{H}) 
+(\chi_{\mu}^{0}\chi_{\nu}^{N}|\dfrac{\chi_{\lambda}^{G}}{\partial{R_{I}}}\chi_{\sigma}^{H}) 
+(\chi_{\mu}^{0}\chi_{\nu}^{N}|\chi_{\lambda}^{G}\dfrac{\chi_{\sigma}^{H}}{\partial{R_{I}}})   ]}
\end{equation}




\section{CONCLUSIONS}

\begin{acknowledgments}

\end{acknowledgments}


\begin{thebibliography} {99}


\bibitem{DFT_KS} W. Kohn and L. Sham, Phys. Rev. {\bf 140}, 1133 (1965).
\bibitem{DFT_HK} P. Hohenberg and W. Kohn, Phys. Rev. B {\bf 136}, 864 (1964).

\bibitem{Parr} R. G. Parr and W. Yang, \emph{Density Functional Theory of Atoms and Molecules }(Oxford University Press, New York,1989)

\end{thebibliography}



\end{document}
